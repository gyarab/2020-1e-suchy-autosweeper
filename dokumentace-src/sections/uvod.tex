\section{Úvod}
\subsection{Zadání projektu}
Autosweeper je variace na známou hru jménem Minesweeper, ve které se hledají miny. Hráč dostane libovolně velké dvourozměrné
pole čtverečků, pod kterými se můžou skrývat miny. Cílem hry je postupně odhalovat čtverečky pod kterými miny nejsou.
Při každém odhalení je hráči ukázáno číslo, které značí počet min v okolních osmi políčkách. Z těchto informací musí hráč 
hádat, jaké políčka může odhalit.

Součástí programu je automat, který hru umí řešit. Hráč si může vybrat mezi třemi herními módy: samostatný, s pomocí a
automatický. Když si hráč vybere samostatnou hru, tak se automat do hry nijak nezapojuje. Při hře s pomocí může hráč
kliknout na tlačítko "analyzovat", které automat spustí a ukáže hráči jaké políčka jsou vyhodnocena jako nebezpečná.
V automatickém módu automat řeší hru celou sám.

Při jakékoliv hře bude také běžet časovač, aby hráč mohl vidět své zlepšení, či zhoršení. Jestliže hru bude řešit
jen automat, tak bude časovač běžet jen při jeho "přemýšlení" neboli vyhodnocování nebezpečnosti políček.

\subsection{Hra Minesweeper a její historie}
Základní část programu Autosweeper je hra Minesweeper, prvně vydaná společností Microsoft v roce 1990 jako součást balíčku 
video her pro operační systém Windows 3.0. Avšak originální nápad této hry náleží video herní firmě Quicksilva a její hře
Mined-Out (1983) pro pročítače ZX Spectrum napsané v jazyce COBOL. Zatímco Mined-Out byla jediná doložená inspirace pro
Minesweeper, první hra s podobnou pointou byla Cube (1973) od Jerimaca Ratliffa. Princip hry Minesweeper byl tedy jeden z
nejstarších. Dokáže se stářím vyrovnat například hře Tetris, nebo Pac-Man. \autocite{minesweeperinfo}
