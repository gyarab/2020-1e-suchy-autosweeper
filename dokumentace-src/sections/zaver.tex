
\section{Instalace}
Instalace Autosweeperu je poměrně jednoduchá. Postup k instalaci:
\begin{enumerate}
    \item Zkompilujte zdrojový kód, nebo si stáhněte zkompilovaný program z GitHubu
    \item Výsledný {\tt .jar} dejte společně s {\tt resources} do jedné složky
    \item Spusťte {\tt .jar} program
\end{enumerate}

\section{Závěr}
Myslím si, že se mi Autosweeper podařil. Uživatelské rozhraní je přesně takové, jaké jsem si představoval. Jediné z čeho jsem lehce
zklamaný je samotný automat. Původně jsem zamýšlel implementovat algoritmus popisovaný ve studii Bercerry \cite{bercerra2015}, ale
nechtěl jsem implementovat věc, které nerozumím. I přes lehké zklamání, že automat není 100\% úspěšný jsem rád, že umí vyřešit i
poměrně těžké hry. Na projektu by se dal zlepšit samotný automat, ale také i celková organizace kódu, jako například render políček,
který je sice velmi automatický, ale občas dělá neplechu.